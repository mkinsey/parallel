\documentclass{article}

\begin{document}
\title{CUDA Program Two}
\author{Michael Kinsey}
\date{9 March 2017}
\maketitle

In this program I present a solution to Problem Set 2 from Udacity's \textit{Intro
to Parallel Programming} course. This program implements a parallel algorithm for
applying a Gaussian blur to images. All modified functions are in the 
\textit{student\_func.cu} file. For this assignment I increased the number of 
threads to 1024 in order to eek more performance out of the program. Execution
time through the Udacity iterface was consistently around 1.66 ms.\\

The first conecptual step in this program is to reorder the data, from contiguous
pixel structures to arrays of RGB values. This is an example of a common 
operation in parallel computing referred to as converting an Array of Structures
to a Structure of Arrays.
We launch a kernel per pixel in the natural way, and call the $separateChannels$
function for each. This separates the image into three distinct data structures,
each holding one color channel.\\

The next step is to apply the filter to each color channel and save the blurred
output into a new array. Now we launch a kernel for each color channel in the 
same way we have done previously, this time calling the $gaussian\_blur$ 
function. This function takes care not to step out of the array bounds, and 
takes the product of each filter pixel with the respective color channel pixel.
The results are summed and then applied (product) to the center pixel. It should
be noted that much of the code for this function was provided in the serialized
version of the function in the \textit{reference\_calc.cpp} file.

Finally, the three color channels must be recombined to form the final blurred 
image. This function has been provided by the course instructors. The device 
allocated memory is then freed and the function terminates.

\end{document}
