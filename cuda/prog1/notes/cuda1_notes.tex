\documentclass{article}

\begin{document}
\title{CUDA Program One}
\author{Michael Kinsey}
\date{7 March 2017}
\maketitle

In this program I present a solution to Problem Set 1 from Udacity's \textit{Intro
to Parallel Programming} course. The program uses a simple equation to convert an
RGBA image to Greyscale. The only modified source file is $student\_func.cu$,
which prompts the student (me) to implement two methods. \\

The first method, $rgba\_to\_greyscale$, is very straightforward. This is the actual
kernel that is ran by each thread. It uses its own x and y indexes as well as the
block's indexes and dimensions to calculate where it is in the RGBA image. As
evident in the source code, both images are represented as large one-dimensional
arrays. After the index in the image is computed, a greyscale value is computed
and the corresponding index in $greyImage$ is updated.\\

The second method is called $your\_rgba\_to\_greyscale$. This function sets up the
kernel launching for the program. Because the actual image is two dimensional, I
set up two-dimensional blocks. In order to maximize the throughput of each block
while maximizing compatability, I chose to use ($22 * 22 $) 484 threads, less
than the maxiumum of 512 for older GPUs. I then specify the number of blocks to
be initialized by dividing the width and height of the image by the values used
to specify the number of threads per block. I add one to each value to ensure
that the remainder of pixels are processed in the highly likely event that the
width and height are not divisible by 22. The kernel launch call then directs
the blocks of threads to calculate $rgba\_to\_greyscale$ using the given parameters. 

\end{document}
