\documentclass{article}

\begin{document}
\title{CUDA Program Six}
\author{Michael Kinsey}
\date{21 April 2017}
\maketitle

In this program I present a solution to Problem Set 6 from Udacity's \textit{Intro
to Parallel Programming} course. This program implements a parallel algorithm for
seamless image cloning. The goal is to blend a source image (in our case a polar bear) 
onto a destination image (a swimming pool). This problem set uses the jacobi method 
for a Poisson equation to calculate the pixel values of the blended image.  All 
modified functions are in the \textit{student\_func.cu} file. Execution time 
through the Udacity interface was around 62 ms.

The first step involves creating a mask for the source image. By the problem 
specification, any non-white pixels should be included in the blended image.
This mask gives us a quick way to check if the pixel is inside of the desired
region of the source image.

Next, two more masks are created based on the previous step. The immediate 
border and the strict interior region are calculated and stored in two arrays. 
The strict interior requires all four of its neighbors to be inside the mask, 
while a border pixel is in the mask itself but has $\geq$ 1 neighbor that is not.
These masks are used in the jacobi iterations to solve for the blended pixel 
values.

The jacobi iterations use a double buffering technique to calculate new values
based on the previous iteration's values. We create two buffers for each color
channel (R, G, B) and initialize them using the values of the source image. This
is effectively our first guess. 

Finally, we may begin an iteration. For each pixel in the strict interior of 
the mask, we will calculate a pixel value based on the pixel's neighbors. If 
the neighbor is in the interior, average the sum of its neighbor's previous 
values with the difference between the source pixel's value and the neighbor's 
source value. If the neighbor is on the border of the image, average the 
neighbor's destination image value with the difference between the source pixel's
value and the neighbor's source value. We continue in this fashion for each 
iteration, updating one buffer's values based on the results in the previous 
buffer, swapping pointers to the buffer at each iteration.

Optimizations to this program included loop unfolding, and combining steps when
possible. For example, the color channels are only separated when the double 
buffers are being initalized. An example of loop unfolding is calculating 
all color channels in the same loop, even though a single loop would reduce
code reduncancy. Further optimizations could be made by precomputing some values,
and only assigning threads to the masked region of the image for the jacobi 
iterations.


\end{document}
